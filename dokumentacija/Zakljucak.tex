\chapter{Zaključak i budući rad}
		
		{Zadatak naše grupe bio je razvoj igre \textit{GeoFighter} u kojoj se igrači međusobno bore kartama. Igrači karte skupljaju na različitim stvarnim lokacijama, a cilj razvoja igre je popunjavanje manjkave baze podataka stvarnih lokacija. Nakon 15 tjedana timskog rada u razvijanju igre, uspješno smo ostvarili zadani cilj. Provedba projekta izvodila se kroz dvije faze.} 
		
		{Prva faza projekta uključivala je okupljanje tima za razvoj igrice, dodjele projektnih zadataka, formiranje podtimova za rad na \textit{frontendu} i \textit{backendu} te dokumentiranje zahtjeva igre. Kvalitetnim dokumentiranjem tijekom prve faze projekta uvelike smo olakšali daljnju implementaciju i rad na samoj igri te osmišljavanju sustava. Izrađeni obrasci i dijagrami (obrasci uporabe, sekvencijski dijagrami, model bazepodataka, dijagram razreda) pomogli su timovima za razvoj \textit{frontenda} i \textit{backenda} koje smo definirali u ranijem dijelu prve faze te su uštedili mnogo vremena tijekom druge faze projekta kada su članovi tima nailazili na nedoumice oko implementacije rješenja.}
			
		{Druga faza projekta uglavnom se temeljila na samostalnom radu članova te na stjecanju novih znanja u slučaju manjka iskustva pri korištenju potrebnih odabranih alata i programskih jezika za izradu implementacijskih rješenja. Također, u drugoj fazi bilo je potrebno dovršiti i dokumentirati ostale UML dijagrame i napisati ostatak dokumentacije projekta. Temeljito dokumentirano željeno ponašanje sustava te dobro izrađeni obrasci i dijagrami tijekom prve faze projekta omogućili su nam da izbjegnemo potencijalne pogreške tijekom razvoja igre koje bi bile vremenski skupe u daljnjoj izradi projekta.}
		
		{Moguće proširenje postojeće inačice sustava je izrada mobilne aplikacije čime bi korisničko iskustvo bilo bogatije i bolje nego ono ostvareno web aplikacijom. Moguće su nadogradnje u obliku organiziranih turnira i dodavanju mogućnosti za proširenje udaljenosti za borbu kao i dodavanje različitih tipova borbi.}
		
		{Sudjelovanje na ovom projektu naučilo nas je koliko znači dobra komunikacija među članovima tima, koliko je nužno međusobno poštivanje i razumijevanje prema kolegama u timu te važnost dobre vremenske organiziranosti i koordiniranosti među članovima. Naposljetku, iako zadovoljni postignutim rezultatima i stečenim iskustvima tijekom rada na projektnom zadatku, svjesni smo potencijalnih nadogradnji kao i onih dijelova koje smo mogli bolje implementirati. Bez obzira na to, puno smo naučili.}\textsc{}
		
		\eject 