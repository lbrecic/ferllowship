\chapter{Arhitektura i dizajn sustava}

		{Prilikom projektiranja samog sustava jedna od važnijih odluka bila je odabir programskih jezika i razvojnih okruženja u kojima ćemo razviti našu aplikaciju. Programski jezici koje smo odabrali su Java sa Spring Boot razvojnim okvirom za \textit{backend} te React Javascript za \textit{frontend}. Odabrana razvojna okruženja su Eclipse za \textit{backend} te Visual Studio Code za \textit{frontend}.}	\\
		
		{Također izbor odgovarajuće arhitekture programske potpore jedan je od najbitnijih koraka u oblikovanju sustava jer ona predstavlja poveznicu između zahtjeva u sustav i same implementacije sustava. Dobra arhitektura znači dobru fleksibilnost sustava, jednostavnu mogućnost nadogradnje i jeftino održavanje.}\\
		
		{Sama arhitektura sustava je vrlo jednostavna, a sastoji se od dvije manje aplikacije: klijenta i poslužitelja. Osnovna prednost modela klijent-poslužitelj je u tome što nije potreban sustav za upravljanje bazom podataka na svakom klijentskom računalu, već se klijent s bazom podataka povezuje preko aplikacije. Time se postiže veća sigurnost i zaštita podataka. Poslužiteljska aplikacija ima pristup bazi podataka u kojoj će se pohranjivati podaci o lokacijama, kartama, borbama, igračima i kartografima. Poslužiteljska aplikacija neće dohvaćati slike i pohranjivati ih u bazu podataka već na servis Cloudinary. Poslužiteljska aplikacija temeljena je na REST principima za izradu web aplikacija te u skladu s time klijentska aplikacija dohvaća podatke iz poslužiteljske i prezentira ih na korisniku razumljiv način.}\\
		
		{Za izradu aplikacije odabrali smo MVC arhitekturu jer omogućava dodatno strukturiranje aplikacije s obzirom na objektno orijentiranu paradigmu što znatno olakšava nezavisni razvoj, ispitivanje i održavanje aplikacije.}		

				
		\section{Baza podataka}
			
			Podatci potrebni za funkcioniranje naše aplikacije pohranjuju se u relacijsku bazu podataka. Osnovni objekt relacijske baze podataka je relacija - imenovana dvodimenzionalna tablica čiji stupci predstavljaju atribute, a retci opisuju entitete baze podataka (retci u relaciji zovu se n-torke).
			Sljedeći entiteti sačinjavaju bazu podataka naše aplikacije:
			\begin{packed_item}
				\item Igrač (\textit{player})
				\item Zabrana pristupa (\textit{ban})
				\item Potvrda registracije (\textit{confirmation})
				\item Kartograf (\textit{mapper})
				\item Administrator (\textit{admin})
				\item Borba (\textit{fight})
				\item Lokacija (\textit{location})
				\item Najkraći put (\textit{path})
				\item Kategorija lokacije (\textit{category})
				\item Karta (\textit{card})
			\end{packed_item}
			
			\subsection{Opis tablica}
			
				\noindent\textbf{player} $($Igrač$)$ - Ovaj entitet sadržava podatke o korisniku. Svaki korisnik je ujedno i igrač. Sadrži atribute: ID korisnika, korisničko ime, hash lozinke, e-mail adresu, fotografiju, bodove igrača, status bana, aktivnost igrača, osposobljenost računa i "experience points". Ovaj entitet je u \textit{One-to-Many} vezi s entitetom card preko atributa player\_id. Ima dvije \textit{One-to-Many} veze s entitetom fight, koje se odnose na borbe u kojima je igrač pobijedio i borbe u kojima je izgubio. Player je u \textit{One-to-One} vezi s entitetom ban i s entitetom confirmation. \\
				
				Atribut username je alternativni ključ entiteta. Atribut ban\_status može poprimiti jednu od sljedećih vrijednosti: 0 - korisnik nije pod banom (nije isključen iz igre$)$, 1 - korisnik je privremeno isključen iz igre, 2 - korisnik je trajno isključen iz igre.
				
				\begin{longtabu} to \textwidth {|X[6, l]|X[7, l]|X[20, l]|}
					
					\hline \multicolumn{3}{|c|}{\textbf{player}}	 \\[3pt] \hline
					\endfirsthead
					
					\hline \multicolumn{3}{|c|}{\textbf{player}}	 \\[3pt] \hline
					\endhead
					
					\hline 
					\endlastfoot
					
					\cellcolor{LightGreen}\textbf{player\_id} & UUID 	&  	jedinstveni brojčani identifikator korisnika 	\\ \hline
					username & VARCHAR(32) 	&  	jedinstveno korisničko ime 	\\ \hline
					pass\_hash & VARCHAR(64)  &   hash lozinke \\ \hline 
					email & VARCHAR(128)  &   jedinstvena e-mail adresa korisnika \\ \hline 
					photo\_link & VARCHAR(200) & fotografija (avatar$)$ korisnika \\ \hline 
					points & INT	&  	broj bodova igrača	\\ \hline
					ban\_status & INT	&  	status o zabranama igrača	\\ \hline 
					activity & BOOLEAN	&  	oznaka je li igrač online	\\ \hline 
					enabled & BOOLEAN	&  	oznaka je li igraču omogućeno korištenje računa nakon registracije	\\ \hline
					experience	& INT	&	 mjera "iskustva" u igri iskazana brojčanom vrijednošću \\ \hline
					
				\end{longtabu}
			
				\noindent\textbf{confirmation} $($Potvrda registracije$)$ - Ovaj entitet sadržava podatke o potvrdi registracije. Sadrži atribute: ID tokena, token i ID korisnika. Ovaj entitet u vezi je \textit{One-to-One} s Player preko jedinstvenog brojčanog identifikatora korisnika.
				
				\begin{longtabu} to \textwidth {|X[6, l]|X[7, l]|X[20, l]|}
					
					\hline \multicolumn{3}{|c|}{\textbf{confirmation}}	 \\[3pt] \hline
					\endfirsthead
					
					\hline \multicolumn{3}{|c|}{\textbf{confirmation}}	 \\[3pt] \hline
					\endhead
					
					\hline 
					\endlastfoot
					
					\cellcolor{LightGreen}\textbf{token\_id} & UUID 	&   jedinstveni brojčani identifikator registracije 	\\ \hline
					token & VARCHAR(255) & token potvrde o registraciji\\ \hline
					\cellcolor{LightBlue}\textit{player\_id} & UUID 	&   jedinstveni brojčani identifikator korisnika 	\\ \hline  
					
				\end{longtabu}
			
				\noindent\textbf{ban} $($Zabrana pristupa$)$ - Ovaj entitet sadržava podatke o igračima kojima je zabranjen pristup aplikaciji. Sadrži atribute kraj zabrane i ID igrača. Ovaj entitet u vezi je \textit{One-to-One} s korisnikom (player$)$ preko ID-a korisnika.
				
				\begin{longtabu} to \textwidth {|X[6, l]|X[6, l]|X[20, l]|}
					
					\hline \multicolumn{3}{|c|}{\textbf{ban}}	 \\[3pt] \hline
					\endfirsthead
					
					\hline \multicolumn{3}{|c|}{\textbf{ban}}	 \\[3pt] \hline
					\endhead
					
					\hline 
					\endlastfoot
					
					\cellcolor{LightBlue}\textit{player\_id} & UUID 	&   jedinstveni brojčani identifikator igrača 	\\ \hline 
					ban\_end & DATE & datum isteka zabrane \\ \hline 
					
				\end{longtabu}
				
				\noindent\textbf{mapper} $($Kartograf$)$ - Entitet mapper nasljeđuje entitet player. Ovaj entitet, uz atribute playera, ima i atribute IBAN i ID photo.
				
				\begin{longtabu} to \textwidth {|X[6, l]|X[7, l]|X[20, l]|}
					
					\hline \multicolumn{3}{|c|}{\textbf{mapper}}	 \\[3pt] \hline
					\endfirsthead
					
					\hline \multicolumn{3}{|c|}{\textbf{mapper}}	 \\[3pt] \hline
					\endhead
					
					\hline 
					\endlastfoot
					
					\cellcolor{LightGreen}\textit{player\_id} & UUID 	&  	jedinstveni brojčani identifikator kartografa 	\\ \hline
					iban & VARCHAR(34)  & IBAN računa za uplatu plaće \\ \hline 
					id\_photo & VARCHAR(200) & fotografija osobne iskaznice \\ \hline 
					
				\end{longtabu}
			
				\noindent\textbf{admin} $($Administrator$)$ - Entitet admin nasljeđuje entitet player. Ovaj entitet ima iste atribute kao i entitet player.
				
				\begin{longtabu} to \textwidth {|X[5, l]|X[7, l]|X[20, l]|}
					
					\hline \multicolumn{3}{|c|}{\textbf{admin}}	 \\[3pt] \hline
					\endfirsthead
					
					\hline \multicolumn{3}{|c|}{\textbf{admin}}	 \\[3pt] \hline
					\endhead
					
					\hline 
					\endlastfoot
					
					\cellcolor{LightGreen}\textit{player\_id} & UUID 	&  	jedinstveni brojčani identifikator administratora 	\\ \hline
					iban & VARCHAR(34)  & IBAN računa za uplatu plaće \\ \hline 
					id\_photo & VARCHAR(200) & fotografija osobne iskaznice \\ \hline  
					
				\end{longtabu}
			
				\noindent\textbf{fight} $($Borba$)$ - Ovaj entitet sadržava podatke o održanim borbama između igrača. Sadrži atribute: ID borbe, trenutak početka borbe, vrijeme trajanja borbe, ID igrača koji je pobijedio i ID igrača koji je izgubio. Ovaj entitet ima dvije \textit{Many-to-One} veze s entitetom player preko identifikatora pobjednika i gubitnika.
				
				\begin{longtabu} to \textwidth {|X[4, l]|X[6, l]|X[22, l]|}
					
					\hline \multicolumn{3}{|c|}{\textbf{fight}}	 \\[3pt] \hline
					\endfirsthead
					
					\hline \multicolumn{3}{|c|}{\textbf{fight}}	 \\[3pt] \hline
					\endhead
					
					\hline 
					\endlastfoot
					
					\cellcolor{LightGreen}\textbf{fight\_id} & UUID	&  	jedinstveni brojčani identifikator borbe 	\\ \hline
					start & TIMESTAMP  &   trenutak početka borbe \\ \hline 
					duration & INTERVAL	&  	trajanje borbe	\\ \hline 
					\cellcolor{LightBlue}\textit{winner}	& UUID & jedinstveni brojčani identifikator pobjednika borbe  	\\ \hline 
					\cellcolor{LightBlue}\textit{loser}	& UUID & jedinstveni brojčani identifikator gubitnika borbe  	\\ \hline 
					
				\end{longtabu}

				{\noindent\textbf{location} $($Lokacija$)$ - Ovaj entitet sadržava sve važne informacije o lokacijama na kojima igrač može sakupiti karte. Sadrži atribute: ID lokacije, naziv lokacije, fotografiju lokacije i ID kategorije. Ovaj entitet u vezi je \textit{Many-to-One} s Category preko ID kategorije te je u vezi \textit{One-to-One} s Path preko ID lokacije. \\
					
				Atribut location\_status može poprimiti jednu od sljedećih vrijednosti: 0 - odobrena, 1 - odbijena, 2 - čeka odobrenje kartografa, 3 - potreban izlazak na teren i pomniji pregled}
				
				\begin{longtabu} to \textwidth {|X[6, l]|X[7, l]|X[20, l]|}
					
					\hline \multicolumn{3}{|c|}{\textbf{location}}	 \\[3pt] \hline
					\endfirsthead
					
					\hline \multicolumn{3}{|c|}{\textbf{location}}	 \\[3pt] \hline
					\endhead
					
					\hline 
					\endlastfoot
					
					\cellcolor{LightGreen}\textbf{location\_id} & UUID	&   jedinstveni brojčani identifikator lokacije	\\ \hline
					location\_name	& VARCHAR(32) &  naziv lokacije 	\\ \hline 
					location\_desc	& TEXT &  opis lokacije 	\\ \hline 
					location\_photo & VARCHAR(200) &  fotografija lokacije \\ \hline 
					location\_status	& INT &  status prihvatljivosti lokacije 	\\ \hline 
					coordinates	& VARCHAR(32) &  koordinate lokacije 	\\ \hline 
					\cellcolor{LightBlue} \textit{category\_id}	& UUID &   jedinstveni brojčani identifikator kategorije	\\ \hline 
					
					
				\end{longtabu}
			
				{\noindent\textbf{path} $($Najkraći put$)$ - Ovaj entitet sadržava sve važne informacije o najkraćem putu do lokacija koje je potrebno provjeriti. Sadrži atribute. Ovaj entitet u vezi je \textit{One-to-One} s Location preko ID lokacije.}
				
				\begin{longtabu} to \textwidth {|X[6, l]|X[7, l]|X[20, l]|}
					
					\hline \multicolumn{3}{|c|}{\textbf{path}}	 \\[3pt] \hline
					\endfirsthead
					
					\hline \multicolumn{3}{|c|}{\textbf{path}}	 \\[3pt] \hline
					\endhead
					
					\hline 
					\endlastfoot
					
					distance & INT &  najkraći put do lokacije \\ \hline
					\cellcolor{LightBlue} \textit{location\_id}	& UUID &   jedinstveni brojčani identifikator lokacije	\\ \hline 
					
					
				\end{longtabu}
			
				{\noindent\textbf{category} $($Kategorija lokacije$)$ - Ovaj entitet sadržava sve važne informacije o kategorijama koje lokacije mogu biti. Sadrži atribute: ID kategorije, naziv kategorije i broj bodova koje kategorija nosi. Ovaj entitet u vezi je \textit{One-to-Many} s Location preko ID kategorije.}
				
				\begin{longtabu} to \textwidth {|X[7, l]|X[7, l]|X[20, l]|}
					
					\hline \multicolumn{3}{|c|}{\textbf{category}}	 \\[3pt] \hline
					\endfirsthead
					
					\hline \multicolumn{3}{|c|}{\textbf{category}}	 \\[3pt] \hline
					\endhead
					
					\hline 
					\endlastfoot
					
					\cellcolor{LightGreen}\textbf{category\_id} & UUID	&   jedinstveni brojčani identifikator kategorije	\\ \hline
					category\_name	& VARCHAR(32) &  naziv kategorije 	\\ \hline 
					category\_points & INT &  bodovna vrijednost kategorije \\ \hline  
					
					
				\end{longtabu}
			
				{\noindent\textbf{card} $($Karta$)$ - Ovaj entitet sadržava sve važne informacije o kartama koje igrači mogu posjedovati. Sadrži atribute: ID karte, broj bodova karte i ID lokacije. Ovaj entitet u vezi je \textit{Many-to-One} s Location preko ID lokacije te \textit{Many-to-One} s korisnikom (player$)$ preko ID-a korisnika.}

				\begin{longtabu} to \textwidth {|X[6, l]|X[6, l]|X[20, l]|}
					
					\hline \multicolumn{3}{|c|}{\textbf{card}}	 \\[3pt] \hline
					\endfirsthead
					
					\hline \multicolumn{3}{|c|}{\textbf{card}}	 \\[3pt] \hline
					\endhead
					
					\hline 
					\endlastfoot
					
					\cellcolor{LightGreen}\textbf{card\_id} & UUID	&   jedinstveni brojčani identifikator karte	\\ \hline 
					card\_points & INT &  bodovna vrijednost karte \\ \hline 
					scale\_factor & INT &  faktor skaliranja bodova karte \\ \hline 
					\cellcolor{LightBlue} \textit{location\_id}	& UUID &   jedinstveni brojčani identifikator lokacije	\\ \hline 
					\cellcolor{LightBlue}\textit{player\_id} & UUID	&   jedinstveni brojčani identifikator korisnika	\\ \hline
					
					
				\end{longtabu}
			
			
			\subsection{Dijagram baze podataka}
				\begin{figure}[H]
					\includegraphics[width=\linewidth, height=14cm]{dijagrami/geofighterdb_diagram}				
					\centering
					\caption{E-R dijagram baze podataka}
					\label{}
				\end{figure}
			
			\eject
			
			
		\section{Dijagram razreda}
		
			{Slike 4.2, 4.3, 4.4 i 4.5 prikazuju razrede \textit{backend} dijela MVC arhitekture. Razredi na slici 4.2 i 4.3 prikazuju razrede Service i razrede Controller s anotacijom @RestController što je specifično za spring boot koji tom anotacijom kombinira anotacije @Controller i @ResponseBody te omogućuje da svaka metoda koja rukuje sa zahtjevima automatski serijalizira povratne vrijednosti objekata u \textit{HttpResponse}. Service razredi služe za modeliranje logike koja se događa nad modelima (npr. slanje mail-a) i služe za odvajanje takve logike od kontrolera čija je zadaća isključivo odgovarati na http zahtjeve (bilo GET, POST, PUT, PATCH ili DELETE).}
			
			\begin{figure}[H]
				\includegraphics[width=17cm, height=10cm]{dijagrami/servicecontroller_diagram}				
				\centering
				\caption{Dijagram razreda - dio trenutnih Controllers i Service razreda}
				\label{}
			\end{figure}
		
			\begin{figure}[H]
				\includegraphics[width=\linewidth, height=14cm]{dijagrami/futureclass_diagram}				
				\centering
				\caption{Dijagram razreda - dio budućih Controller i Service razreda}
				\label{}
			\end{figure}
			
			\begin{figure}[H]
				\includegraphics[width=\linewidth, height=14cm]{dijagrami/daoclass_diagram}				
				\centering
				\caption{Dijagram razreda - dio Data access object razreda}
				\label{}
			\end{figure}
		
			{Model razredi preslikavaju strukturu baze podataka aplikacije. Razred Player predstavlja igrača koji se može registrirati, raspolaže svojim špilom karata koje skuplja te sudjeluje u borbama i otkrivanju novih lokacija. Razred Admin predstavlja administratora te nasljeđuje sve funkcije razreda Player i ima sve ovlasti nad bazom podataka i upravljanja igračima svih razina. Razred Cartograph predstavlja kartografa koji nasljeđuje sve funkcije razreda Player i ima mogućnosti upravljanja svim postojećim i novim lokacijama. Razred Fight predstavlja borbu u kojoj sudjeluju dva igrača. Razred Card predstavlja kartu koja obzirom na kategoriju lokacije kojoj pripada sadrži određen broj bodova koji se koristi u borbama. Razred Location predstavlja lokaciju na kojoj se mogu skupljati karte ukoliko ih kartograf odobri. Razred Category predstavlja kategoriju lokacije prema čijoj se klasifikaciji određuje broj bodova koje lokacije donose.}
			
			\begin{figure}[H]
				\includegraphics[width=\linewidth, height=14cm]{dijagrami/modelclass_diagram}				
				\centering
				\caption{Dijagram razreda - dio Models razreda}
				\label{}
			\end{figure}
			
			\textbf{\textit{dio 2. revizije}}\\			
			
			\textit{Prilikom druge predaje projekta dijagram razreda i opisi moraju odgovarati stvarnom stanju implementacije}
			
			
			
			\eject
		
		\section{Dijagram stanja}
			
			
		%	\textbf{\textit{dio 2. revizije}}\\
			
		%	\textit{Potrebno je priložiti dijagram stanja i opisati ga. Dovoljan je jedan dijagram stanja koji prikazuje \textbf{značajan dio funkcionalnosti} sustava. Na primjer, stanja korisničkog sučelja i tijek korištenja neke ključne funkcionalnosti jesu značajan dio sustava, a registracija i prijava nisu. }
			
		{Dijagram stanja služi za opis diskretnih stanja sustava i prijelaza između tih stanja. Na slici je prikazan dijagram stanja registriranog korisnika (igrača). Nakon prijave igraču se prikazuje početna stranica ("Home") na kojoj može prijeći na stranicu borbe. Odlaskom na tu stranicu igrač se može dopisivati s aktivnim igračima te poslati zahtjeve za borbu. Također, igrač može u padajućem izborniku odabrati stranicu za prikaz: profila ("Profile"), mape i lokacija ("Map"), kolekciju karata ("Deck"), globalne statistike ("Global statistics"), te stranica "Help". U padajućem izborniku može se i vratiti na početnu stranicu te iz bilo koje stranice se može odjaviti. Na stranici profila igrač može mijenjati osobne podatke, pregledati vlastitu statistiku i kolekciju karata. Također može ispuniti i poslati zahtjev da postane kartograf. Na stranici "Map" igrač vidi dostupne lokacije (i njihove karakteristike) te može predati zahtjev za novu lokaciju. Na stranici za prikaz karata igrač vidi kolekciju karta, a na stranici globalne statistike može vidjeti svoj rang na globalnoj statistici. }
			\begin{figure}[H]
				\includegraphics[width=18cm, height=12cm]{dijagrami/dijagram stanja-igrač}				
				\centering
				\caption{Dijagram stanja}
				\label{}
			\end{figure}
			\eject 
		
		\section{Dijagram aktivnosti}
			
			\textbf{\textit{dio 2. revizije}}\\
			
			 \textit{Potrebno je priložiti dijagram aktivnosti s pripadajućim opisom. Dijagram aktivnosti treba prikazivati značajan dio sustava.}
			
			\eject
		\section{Dijagram komponenti}
		
			\textbf{\textit{dio 2. revizije}}\\
		
			 \textit{Potrebno je priložiti dijagram komponenti s pripadajućim opisom. Dijagram komponenti treba prikazivati strukturu cijele aplikacije.}