\chapter{Specifikacija programske potpore}
		
	\section{Funkcionalni zahtjevi}
			
			\textbf{\textit{dio 1. revizije}}\\
			
			\textit{Navesti \textbf{dionike} koji imaju \textbf{interes u ovom sustavu} ili  \textbf{su nositelji odgovornosti}. To su prije svega korisnici, ali i administratori sustava, naručitelji, razvojni tim.}\\
				
			\textit{Navesti \textbf{aktore} koji izravno \textbf{koriste} ili \textbf{komuniciraju sa sustavom}. Oni mogu imati inicijatorsku ulogu, tj. započinju određene procese u sustavu ili samo sudioničku ulogu, tj. obavljaju određeni posao. Za svakog aktora navesti funkcionalne zahtjeve koji se na njega odnose.}\\
			
			
			\noindent \textbf{Dionici:}
			
			\begin{packed_enum}
				
				\item Igrači
				\item Administrator			
				\item Razvojni tim
				
			\end{packed_enum}
			
			\noindent \textbf{Aktori i njihovi funkcionalni zahtjevi:}
			
			
			\begin{packed_enum}
				\item  \underbar{Neregistrirani/neprijavljeni korisnik (inicijator) može:}
				
				\begin{packed_enum}
					
					\item Obaviti registraciju unosom korisničkog imena, fotografije, e-mail adrese i lozinke
					\item Ukoliko je korisnik već registriran u sustavu, mora se prijaviti koristeći korisničko ime ili e-mail adresu i lozinku
				\end{packed_enum}
			
				\item  \underbar{Igrač (inicijator) može:}
					
					\begin{packed_enum}
					
						\item Pregledavati i mijenjati svoje korisničke podatke
						\item Izbrisati svoj korisnički račun
						\item Vidjeti kartu na kojoj su označene lokacije na kojima se mogu skupiti karte te njegova lokacija
						\item Informacije o lokacijama (naziv, opis, fotografiju, kategoriju, jačinu karte)
						\item Kolekciju svojih karata
						\item ....jos treba
					
					\end{packed_enum}
			
				\item  \underbar{Administrator (inicijator) može:}
				
				\begin{packed_enum}
					
					\item Vidjeti popis svih registriranih korisnika i njihovih osobnih podataka
					\item Brisati korisnike
					\item  ...treba jos
					
				\end{packed_enum}
			
				\item  \underbar{Aktor 2 (sudionik) može:}
				
				\begin{packed_enum}
					
					\item funkcionalnost 1
					\item funkcionalnost 2
					
				\end{packed_enum}
			\end{packed_enum}
			
			\eject 
			
			
				
			\subsection{Obrasci uporabe}
		
				\subsubsection{Opis obrazaca uporabe}
					
					\noindent \underbar{\textbf{UC1 -Registracija}}
					\begin{packed_item}
	
						\item \textbf{Glavni sudionik: }Korisnik
						\item  \textbf{Cilj:} Stvoriti korisnički račun za pristup sustavu
						\item  \textbf{Sudionici:} Baza podataka
						\item  \textbf{Preduvjet:} -
						\item  \textbf{Opis osnovnog tijeka:}
						
						\item[] \begin{packed_enum}
	
							\item Korisnik odabire opciju za registraciju
							\item Korisnik unosi potrebne korisničke podatke(korisničko ime, e-mail, lozinka)
							\item Korisnik prima obavijest o uspješnoj registraciji
							\item Nakon registracije korisnik automatski prijavljen u sustav
						\end{packed_enum}
						
						\item  \textbf{Opis mogućih odstupanja:}
						
						\item[] \begin{packed_item}
	
							\item[2.a] Odabir već zauzetog korisničkog imena i/ili e-maila ili pružanje neispravnog e-maila
							\item[] \begin{packed_enum}
								
								\item Sustav obavještava korisnika o neuspjelom upisu i vraća ga na stranicu za registraciju 
								\item Korisnik mijenja potrebne podatke te završava unos ili odustaje od registracije
								
							\end{packed_enum}
							\item[2.b] Odabir slabe lozinke(za jaku lozinku obavezno barem 8 znakova, 1 veliko slovo, 1 broj)
							\item[] \begin{packed_enum}
								\item Sustav obavještava korisnika da je lozinka slaba i vraća ga na stranicu za registraciju
								\item Korisnik odabire jaču lozinku te završava unos ili odustaje od registracije 
								
							\end{packed_enum}
								
						\end{packed_item}
					\end{packed_item}
				
					\noindent \underbar{\textbf{UC2 -Prijava u sustav}}
					\begin{packed_item}
					
					\item \textbf{Glavni sudionik: }Klijent
					\item  \textbf{Cilj:} Dobiti pristup korisničkom sučelju
					\item  \textbf{Sudionici:} Baza podataka
					\item  \textbf{Preduvjet:} Registracija
					\item  \textbf{Opis osnovnog tijeka:}
					
					\item[] \begin{packed_enum}
						
						\item Unos korisničkog imena ili e-maila i lozinke
						\item Potvrda o ispravnosti unesenih podataka
						\item Pristup korisničkim funkcijama
					\end{packed_enum}
					
					\item  \textbf{Opis mogućih odstupanja:}
					
					\item[] \begin{packed_item}
						
						\item[2.a] Unos neispravnog korisničkog imena, e-maila ili lozinke
						\item[] \begin{packed_enum}
							
							\item Sustav obavještava korisnika o neispravnom upisu i vraća ga na stranicu za prijavu
							
						\end{packed_enum}
						
					\end{packed_item}
					\end{packed_item}
				
					\noindent \underbar{\textbf{UC3 -Pregled osobnih podataka}}
					\begin{packed_item}
					
					\item \textbf{Glavni sudionik: }Klijent
					\item  \textbf{Cilj:} Pregledati osobne podatke
					\item  \textbf{Sudionici:} Baza podataka
					\item  \textbf{Preduvjet:} Klijent je prijavljen
					\item  \textbf{Opis osnovnog tijeka:}
					
					\item[] \begin{packed_enum}
						
						\item Korisnik odabire opciju "Osobni podatci"
						\item Aplikacija prikazuje osobne podatke korisnika
					\end{packed_enum}
					\end{packed_item}
			
					\noindent \underbar{\textbf{UC4 -Promjena osobnih podataka}}
					\begin{packed_item}
					
					\item \textbf{Glavni sudionik: }Klijent
					\item  \textbf{Cilj:} Promjeniti osobne podatke
					\item  \textbf{Sudionici:} Baza podataka
					\item  \textbf{Preduvjet:} Klijent je prijavljen
					\item  \textbf{Opis osnovnog tijeka:}
					
					\item[] \begin{packed_enum}
						
						\item Korisnik odabire opciju za promjenu podataka
						\item Korisnik mijenja svoje osobne podatke
						\item Korisnik sprema promjene
						\item Baza podataka se ažurira
					\end{packed_enum}
					
					\item  \textbf{Opis mogućih odstupanja:}
					
					\item[] \begin{packed_item}
						
						\item[2.a] Odabir već zauzetog korisničkog imena i/ili e-maila
						\item[] \begin{packed_enum}
							
							\item Sustav obavještava korisnika da je korisničko ime i/ili e-mail već zauzeto i traži ponovni unos
							\item Korisnik unosi novo korisničko ime i/ili e-maila ili odustaje je promjene osobnih podataka
							
						\end{packed_enum}
						\item[2.b] Korisnik odabire "slabu" zamjensku lozinku
						\item[] \begin{packed_enum}
							\item Sustav obavještava korisnika da je odabrao "slabu" lozinku i traži ponovni unos
							\item Korisnik unosi novu lozinku koja je dovoljno "jaka" ili odustaje od promjene lozinke
						\end{packed_enum}
						\item[2.c] Korisnik promijeni svoje osobne podatke, ali ne odabere opciju "Spremi promjenu"
						\item[] \begin{packed_enum}
							\item Sustav obavještava korisnika da nije spremio podatke prije izlaska iz prozora
						\end{packed_enum}
						
					\end{packed_item}
					\end{packed_item}
					
					\noindent \underbar{\textbf{UC -Brisanje korisničkog računa}}
					\begin{packed_item}
						
						\item \textbf{Glavni sudionik: }Klijent
						\item  \textbf{Cilj:} Izbrisati svoj korisnički račun
						\item  \textbf{Sudionici:} Baza podataka
						\item  \textbf{Preduvjet:} Klijent je prijavljen
						\item  \textbf{Opis osnovnog tijeka:}
						
						\item[] \begin{packed_enum}
							
							\item Korisnik otvara stranicu s osobnim podacima
							\item Korisnik briše račun
							\item Sustav traži potvrdu brisanja korisničkog računa
							\item Korisnik odobrava brisanje
							\item Korisnički račun se izbriše iz baze podataka
							\item Otvara se stranica za registraciju
						\end{packed_enum}
						
					\end{packed_item}
				
					\noindent \underbar{\textbf{UC6 -Pregled skupljenih}}
					\begin{packed_item}
						
						\item \textbf{Glavni sudionik: }Klijent
						\item  \textbf{Cilj:} Pregled skupljenih karata klijenta
						\item  \textbf{Sudionici:} Baza podataka
						\item  \textbf{Preduvjet:} Klijent je prijavljen i skupio barem jednu kartu
						\item  \textbf{Opis osnovnog tijeka:}
						
						\item[] \begin{packed_enum}
							
							\item Korisnik odabire opciju za pregled skupljenih karata
							\item Otvara se stranica sa skupljenim kartama korisnika
						\end{packed_enum}
						
						\item  \textbf{Opis mogućih odstupanja:}
						
						\item[] \begin{packed_item}
							
							\item[2.a] Korisnik nije skupio niti jednu kartu
							\item[] \begin{packed_enum}
								
								\item Sustav obavještava korisnika da mora skupiti barem jednu kartu da moze pristupiti pregledu i vraća ga na kartu
								
							\end{packed_enum}
						\end{packed_item}
					\end{packed_item}
				
					\noindent \underbar{\textbf{UC7 - Pregled liste lokacija}}
					\begin{packed_item}
						
						\item \textbf{Glavni sudionik: }Klijent
						\item  \textbf{Cilj:} Pregled lokacija koje klijent može osvojiti
						\item  \textbf{Sudionici:} Baza podataka
						\item  \textbf{Preduvjet:} Klijent je prijavljen
						\item  \textbf{Opis osnovnog tijeka:}
						
						\item[] \begin{packed_enum}
							
							\item Korisnik odabire opciju pregled liste lokacija
							\item Otvara se stranica s pregledom svih mogućih lokacija
						\end{packed_enum}
					\end{packed_item}
				
					\noindent \underbar{\textbf{UC8 - Karta}}
					\begin{packed_item}
						
						\item \textbf{Glavni sudionik: }Klijent
						\item  \textbf{Cilj:} Klijent može vidjeti lokacije i ostale igrače u svojoj okolini
						\item  \textbf{Sudionici:} Baza podataka
						\item  \textbf{Preduvjet:} Klijent je prijavljen i ima uključenu lokaciju
						\item  \textbf{Opis osnovnog tijeka:}
						
						\item[] \begin{packed_enum}
							
							\item Korisnik odabire opciju za prikaz karte
							\item Otvara se karta s prikazom lokacija i igrača u blizini korisnika
						\end{packed_enum}
						
						\item  \textbf{Opis mogućih odstupanja:}
						
						\item[] \begin{packed_item}
							
							\item[2.a] Klijent nema uključenu lokaciju
							\item[] \begin{packed_enum}
								
								\item Sustav obavještava korisnika da uključi lokaciju
							\end{packed_enum}
						\end{packed_item}
					\end{packed_item}
					
					\noindent \underbar{\textbf{UC9 - Borba}}
					\begin{packed_item}
						
						\item \textbf{Glavni sudionik: }Klijent
						\item  \textbf{Cilj:} Borba igrača za najbolji rang
						\item  \textbf{Sudionici:} Baza podataka
						\item  \textbf{Preduvjet:} Klijent je prijavljen
						\item  \textbf{Opis osnovnog tijeka:}
						
						\item[] \begin{packed_enum}
							
							\item Korisnik odabire igrača s kojim se želi boriti
							\item Izazvani igrač prihvaća ili odbija borbu
							\item Ako je igrač prihvatio borbu, borba počinje
						\end{packed_enum}
					\end{packed_item}
				
					\noindent \underbar{\textbf{UC10 - Pregled globalnog ranga igrača}}
					\begin{packed_item}
						
						\item \textbf{Glavni sudionik: }Klijent
						\item  \textbf{Cilj:} Pregled globalnog ranga igrača
						\item  \textbf{Sudionici:} Baza podataka
						\item  \textbf{Preduvjet:} Klijent je prijavljen
						\item  \textbf{Opis osnovnog tijeka:}
						
						\item[] \begin{packed_enum}
							
							\item Korisnik odabire opciju za pregled globalnog ranga igrača
							\item Otvara se stranica s globalnim rangom igrača
						\end{packed_enum}
						
					\end{packed_item}
				
					\noindent \underbar{\textbf{UC11 - Pregled odigranih borbi}}
					\begin{packed_item}
						
						\item \textbf{Glavni sudionik: }Klijent
						\item  \textbf{Cilj:} Pregled borbi u kojima je igrač sudjelovao
						\item  \textbf{Sudionici:} Baza podataka
						\item  \textbf{Preduvjet:} Klijent je prijavljen i odigrao barem jednu borbu
						\item  \textbf{Opis osnovnog tijeka:}
						
						\item[] \begin{packed_enum}
							
							\item Korisnik odabire opciju za pregled odigranih borbi
							\item Otvara se stranica s odigranim borbama
						\end{packed_enum}
						
						\item  \textbf{Opis mogućih odstupanja:}
						
						\item[] \begin{packed_item}
							
							\item[2.a] Korisnik nije sudjelovao u niti jednoj borbi
							\item[] \begin{packed_enum}
								
								\item Sustav obavještava korisnika da nije sudjelovao u brobi
							\end{packed_enum}
						\end{packed_item}
					\end{packed_item}
				
					\noindent \underbar{\textbf{UC12 - Pregled profila ostalih igrača}}
					\begin{packed_item}
						
						\item \textbf{Glavni sudionik: }Klijent
						\item  \textbf{Cilj:} Pregled profila ostalih igrača
						\item  \textbf{Sudionici:} Baza podataka
						\item  \textbf{Preduvjet:} Klijent je prijavljen
						\item  \textbf{Opis osnovnog tijeka:}
						
						\item[] \begin{packed_enum}
							
							\item Korisnik odabire profil drugog igrača
							\item Otvara se profil drugog igrača
						\end{packed_enum}
					\end{packed_item}
				
					\noindent \underbar{\textbf{UC13 - Dodavanje lokacije}}
					\begin{packed_item}
						
						\item \textbf{Glavni sudionik: }Klijent, kartograf
						\item  \textbf{Cilj:} Dodati novu lokaciju
						\item  \textbf{Sudionici:} Baza podataka
						\item  \textbf{Preduvjet:} Klijent je prijavljen
						\item  \textbf{Opis osnovnog tijeka:}
						
						\item[] \begin{packed_enum}
							
							\item Korisnik odabire lokaciju koju želi dodati
							\item Korisnik dodaje karakteristike lokacije
							\item Kartograf potvrđuje ispravnost lokacije
							\item Baza podataka se ažurira
						\end{packed_enum}
					\end{packed_item}
				
					\noindent \underbar{\textbf{UC14 - Brisanje lokacije}}
					\begin{packed_item}
						
						\item \textbf{Glavni sudionik: }Kartograf
						\item  \textbf{Cilj:} Obrisati nepostojeću/neispravnu lokaciju
						\item  \textbf{Sudionici:} Baza podataka
						\item  \textbf{Preduvjet:} -
						\item  \textbf{Opis osnovnog tijeka:}
						
						\item[] \begin{packed_enum}
							
							\item Kartograf odabire lokaciju koju želi obrisati
							\item Baza podataka se ažurira
						\end{packed_enum}
					\end{packed_item}
				
					\noindent \underbar{\textbf{UC15 - Promjena podataka lokacije}}
					\begin{packed_item}
						
						\item \textbf{Glavni sudionik: }Kartograf
						\item  \textbf{Cilj:} Promjeniti zastarijele podatke o lokaciji
						\item  \textbf{Sudionici:} Baza podataka
						\item  \textbf{Preduvjet:} -
						\item  \textbf{Opis osnovnog tijeka:}
						
						\item[] \begin{packed_enum}
							
							\item Kartograf odabire lokaciju za koju želi promijeniti podatke
							\item Kartograf mijenja podatke
							\item Kartograf sprema promjene
							\item Baza podataka se ažurira
						\end{packed_enum}
						
						\item  \textbf{Opis mogućih odstupanja:}
						
						\item[] \begin{packed_item}
							
							\item[3.a] Kartograf promijeni podatke lokacije, ali ne odabere opciju "Spremi promjenu"
							\item[] \begin{packed_enum}
								\item Sustav obavještava kartografa da nije spremio podatke prije izlaska iz prozora
							\end{packed_enum}
							
						\end{packed_item}
					\end{packed_item}
					
					\noindent \underbar{\textbf{UC16 - Pregled korisnika}}
					\begin{packed_item}
						
						\item \textbf{Glavni sudionik: }Administrator
						\item  \textbf{Cilj:} Pregledati registrirane korisnike
						\item  \textbf{Sudionici:} Baza podataka
						\item  \textbf{Preduvjet:} Korisnik je prijavljen i dodijeljena su mu prava administratora
						\item  \textbf{Opis osnovnog tijeka:}
						
						\item[] \begin{packed_enum}
							
							\item Administrator odabire opciju pregledavanja korisnika
							\item Prikaže se lista svih ispravno registriranih korisnika s osobnim podacima
						\end{packed_enum}
					\end{packed_item}
				
					\noindent \underbar{\textbf{UC17 - Brisanje korisnika}}
					\begin{packed_item}
						
						\item \textbf{Glavni sudionik: }Administrator
						\item  \textbf{Cilj:} Obrisati korisnika
						\item  \textbf{Sudionici:} Baza podataka
						\item  \textbf{Preduvjet:} Korisnik je prijavljen i dodijeljena su mu prava administratora
						\item  \textbf{Opis osnovnog tijeka:}
						
						\item[] \begin{packed_enum}
							
							\item Administrator odabire opciju uklanjanja korisnika
							\item Administrator pronalazi željenog korisnika
							\item Administrator uklanja željenog korisnika i njegove podatke iz baze podataka
							\item Baza podataka se ažurira
						\end{packed_enum}
					\end{packed_item}
				
					\noindent \underbar{\textbf{UC18 - Promjena prava pristupa}}
					\begin{packed_item}
						
						\item \textbf{Glavni sudionik: }Administrator
						\item  \textbf{Cilj:} Promijeniti razinu pristupa korisnika (klijent, kartograf, administrator)
						\item  \textbf{Sudionici:} Baza podataka
						\item  \textbf{Preduvjet:} Korisnik je prijavljen i dodijeljena su mu prava administratora
						\item  \textbf{Opis osnovnog tijeka:}
						
						\item[] \begin{packed_enum}
							
							\item Administrator odabire opciju promjene prava pristupa
							\item Administrator pronalazi željenog korisnika
							\item Administrator mijenja razinu pristupa željenom korisniku
						\end{packed_enum}
					\end{packed_item}
					
					
				\subsubsection{Dijagrami obrazaca uporabe}
					
					\textit{Prikazati odnos aktora i obrazaca uporabe odgovarajućim UML dijagramom. Nije nužno nacrtati sve na jednom dijagramu. Modelirati po razinama apstrakcije i skupovima srodnih funkcionalnosti.}
				\eject		
				
			\subsection{Sekvencijski dijagrami}
				
				\textbf{\textit{dio 1. revizije}}\\
				
				\textit{Nacrtati sekvencijske dijagrame koji modeliraju najvažnije dijelove sustava (max. 4 dijagrama). Ukoliko postoji nedoumica oko odabira, razjasniti s asistentom. Uz svaki dijagram napisati detaljni opis dijagrama.}
				\eject
	
		\section{Ostali zahtjevi}
		
			\textbf{\textit{dio 1. revizije}}\\
		 
			 \textit{Nefunkcionalni zahtjevi i zahtjevi domene primjene dopunjuju funkcionalne zahtjeve. Oni opisuju \textbf{kako se sustav treba ponašati} i koja \textbf{ograničenja} treba poštivati (performanse, korisničko iskustvo, pouzdanost, standardi kvalitete, sigurnost...). Primjeri takvih zahtjeva u Vašem projektu mogu biti: podržani jezici korisničkog sučelja, vrijeme odziva, najveći mogući podržani broj korisnika, podržane web/mobilne platforme, razina zaštite (protokoli komunikacije, kriptiranje...)... Svaki takav zahtjev potrebno je navesti u jednoj ili dvije rečenice.}
			 
			 
			 
	