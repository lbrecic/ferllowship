\chapter{Specifikacija programske potpore}
		
	\section{Funkcionalni zahtjevi}			
			
			\noindent \textbf{Dionici:}
			
			\begin{packed_enum}
				
				\item Igrači
				\item Kartografi
				\item Administrator			
				\item Razvojni tim
				
			\end{packed_enum}
			
			\noindent \textbf{Aktori i njihovi funkcionalni zahtjevi:}
			
			
			\begin{packed_enum}
				\item  \underbar{Neregistrirani/neprijavljeni korisnik (inicijator) može:}
				
				\begin{packed_enum}
					
					\item obaviti registraciju:
						\begin{packed_enum}
							\item kao igrač  unosom korisničkog imena, fotografije, e-mail adrese i lozinke
							\item kao kartograf unosom korisničkog imena, fotografije, e-mail adrese, lozinke, IBAN-a te fotografije osobne iskaznice
						\end{packed_enum}
					\item ukoliko je korisnik već registriran u sustavu, mora se prijaviti koristeći korisničko ime ili e-mail adresu i lozinku
				\end{packed_enum}
			
				\item  \underbar{Igrač (inicijator) može:}
					
					\begin{packed_enum}
					
						\item pregledavati i mijenjati svoje korisničke podatke
						\item izbrisati svoj korisnički račun
						\item vidjeti kartu na kojoj su označene lokacije na kojima se mogu skupiti karte te njegova lokacija
						\item vidjeti informacije o lokacijama (naziv, opis, fotografiju, kategoriju i jačinu karte te lokacije)
						\item vidjeti kolekciju svojih karata
						\item vidjeti popis ostalih aktivnih igrača koji se nalaze u krugu od 50km i s njima ući u borbu
						\item prijaviti novu lokaciju
						\item vidjeti globalnu statistiku odigranih borbi i sakupljenih lokacija
						\item vidjeti poredak svih igrača
						\item vidjeti profil drugog igrača (njegove karte, rang na globalnoj ljestvici i statistike vezane uz zadnjih 10 borbi s drugim igračima)
					
					\end{packed_enum}
				
				\item  \underbar{Kartograf (inicijator) može:}
				
				\begin{packed_enum}
					
					\item vidjeti kartu sa:
					\begin{packed_enum}
						
						\item prijavljenim lokacijama za unos u bazu podataka
						\item prikazanim najkraćim putem do lokacije koju treba provjeriti na terenu
						\item već unesenim lokacijama u bazi podataka
					
					\end{packed_enum}
					
					\item odbiti, potvrditi, urediti ili označiti da je potrebna potvrda s terena za prijavljenu lokaciju
					\item dodavati lokacije u bazu podataka
					\item vidjeti i izmjeniti svoje osobne podatke
					
				\end{packed_enum}
			
				\item  \underbar{Administrator (inicijator) može:}
				
				\begin{packed_enum}
					
					\item vidjeti i uređivati popis svih korisnika i njihovih osobnih podataka
					\item dodjeliti igračima privremeno isključenje iz igre
					\item vidjeti i uređivati postojeće lokacije
					\item vidjeti globalnu statistiku odigranih borbi i sakupljenih lokacija
					\item vidjeti poredak svih igrača
					
				\end{packed_enum}
			
				\item  \underbar{Baza podataka (sudionik):}
				
				\begin{packed_enum}
					
					\item pohranjuje sve podatke o korisnicima i njihovim ovlastima
					\item  pohranjuje sve podatke o kartama i lokacijama
					\item  pohranjuje sve podatke o borbama i rang listama igrača
					
				\end{packed_enum}
			\end{packed_enum}
			
			\eject 
			
			
				
			\subsection{Obrasci uporabe}
				
				\textbf{\textit{dio 1. revizije}}
				
				\subsubsection{Opis obrazaca uporabe}
					\textit{Funkcionalne zahtjeve razraditi u obliku obrazaca uporabe. Svaki obrazac je potrebno razraditi prema donjem predlošku. Ukoliko u nekom koraku može doći do odstupanja, potrebno je to odstupanje opisati i po mogućnosti ponuditi rješenje kojim bi se tijek obrasca vratio na osnovni tijek.}\\
					

					\noindent \underbar{\textbf{UC$<$broj obrasca$>$ -$<$ime obrasca$>$}}
					\begin{packed_item}
	
						\item \textbf{Glavni sudionik: }$<$sudionik$>$
						\item  \textbf{Cilj:} $<$cilj$>$
						\item  \textbf{Sudionici:} $<$sudionici$>$
						\item  \textbf{Preduvjet:} $<$preduvjet$>$
						\item  \textbf{Opis osnovnog tijeka:}
						
						\item[] \begin{packed_enum}
	
							\item $<$opis korak jedan$>$
							\item $<$opis korak dva$>$
							\item $<$opis korak tri$>$
							\item $<$opis korak četiri$>$
							\item $<$opis korak pet$>$
						\end{packed_enum}
						
						\item  \textbf{Opis mogućih odstupanja:}
						
						\item[] \begin{packed_item}
	
							\item[2.a] $<$opis mogućeg scenarija odstupanja u koraku 2$>$
							\item[] \begin{packed_enum}
								
								\item $<$opis rješenja mogućeg scenarija korak 1$>$
								\item $<$opis rješenja mogućeg scenarija korak 2$>$
								
							\end{packed_enum}
							\item[2.b] $<$opis mogućeg scenarija odstupanja u koraku 2$>$
							\item[3.a] $<$opis mogućeg scenarija odstupanja  u koraku 3$>$
							
						\end{packed_item}
					\end{packed_item}
				
					
				\subsubsection{Dijagrami obrazaca uporabe}
					
					\textit{Prikazati odnos aktora i obrazaca uporabe odgovarajućim UML dijagramom. Nije nužno nacrtati sve na jednom dijagramu. Modelirati po razinama apstrakcije i skupovima srodnih funkcionalnosti.}
				\eject		
				
			\subsection{Sekvencijski dijagrami}
				
				\textbf{\textit{dio 1. revizije}}\\
				
				\textit{Nacrtati sekvencijske dijagrame koji modeliraju najvažnije dijelove sustava (max. 4 dijagrama). Ukoliko postoji nedoumica oko odabira, razjasniti s asistentom. Uz svaki dijagram napisati detaljni opis dijagrama.}
				\eject
	
		\section{Ostali zahtjevi}
		
			\textbf{\textit{dio 1. revizije}}\\
		 
			 \textit{Nefunkcionalni zahtjevi i zahtjevi domene primjene dopunjuju funkcionalne zahtjeve. Oni opisuju \textbf{kako se sustav treba ponašati} i koja \textbf{ograničenja} treba poštivati (performanse, korisničko iskustvo, pouzdanost, standardi kvalitete, sigurnost...). Primjeri takvih zahtjeva u Vašem projektu mogu biti: podržani jezici korisničkog sučelja, vrijeme odziva, najveći mogući podržani broj korisnika, podržane web/mobilne platforme, razina zaštite (protokoli komunikacije, kriptiranje...)... Svaki takav zahtjev potrebno je navesti u jednoj ili dvije rečenice.}
			 
			 
			 
	