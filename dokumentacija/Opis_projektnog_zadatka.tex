\chapter{Opis projektnog zadatka}
		
		\textbf{\textit{\large 2.1 Uvod}}\\
		
		{Kroz ovaj je projekt potrebno razviti programsku potporu za igru  \emph{GeoFighter}. \emph{GeoFighter} je web-aplikacija kojom se u obliku karata određenih jačina evidentiraju stvarne lokacije koje su korisnici (č. igrači) posjetili. Skupljenim se kartama tada geografski bliski igrači mogu međusobno boriti. Pobjednik borbe dobiva određeni broj bodova koji se oduzimaju gubitniku te se promjenom bodova mijenja i globalni rang igrača. Cilj igrača je posjetiti što veći broj mjesta, skupiti što više karata i postići što bolji rang. } \\ \\
		
		
		\textbf{\textit{\large 2.2 Razrada }}\\ 
		
		\textbf{\textit{2.2.1 Registracija i prijava }}\\
		
		{Prije korištenja same aplikacije, ukoliko je korisnik novi igrač, dužan je najprije registrirati se. Za registraciju potrebni su:
		}
	
		\begin{packed_item}
			\item {korisničko ime,}
			\item {fotografija,}
			\item {lozinka,}
			\item {e-mail adresa.}
		\end{packed_item}
		
		{Validacijom podataka utvrđuje se ispravnost unosa podataka te se provjerava postoji li već korisnik s istim korisničkim imenom ili e-mail adresom. Kada registracije bude gotova, korisniku se na navedenu adresu šalje poveznica putem koje potvrđuje svoj račun. Tek nakon što je račun potvrđen, omogućuje se ulazak u sâm igrački sustav.}
		{Ako je igrač već prethodno registriran, on u sustav ulazi upisujući korisničko ime ili e-mail adresu, i lozinku.}\\ \\
		
		\textbf{\textit{2.2.2 Korištenje sustava}}\\
		
		\textbf{\textit{\small2.2.2.1 Početni zaslon i mogućnosti}}\\
		
		
		{Pri ulasku u sustav korisniku se na karti prikazuje njegova lokacija zajedno s obližnjim lokacijama. Pritiskom na određenu lokaciju otvara se prozor s atributima poput naziva lokacije, opisa, fotografije, kategorije i jačina karte te lokacije. Kategorije lokacija uključuju naseljena mjesta, vrhove planina i gora, umjetničke instalacije... Na zaslonu je u lijevom gornjem kutu simbol za padajući izbornik. Pritiskom na taj simbol, izbornik se otvara na vrhu ili sa strane, ovisno o veličini ekrana, te prikazuje simbole koji označavaju:  }
		
			\begin{packed_item}
			\item {Profil korisnika - otvara novi prozor u kojem se prikazuju podatci o korisniku(korisničko ime, fotografija, lozinka, e-mail adresa). Za fotografiju i korisničko ime sa strane nudi se mogućnost uređivanja.}\\
			\item {Kolekciju korisnikovih karata - otvara novi prozor unutar kojega su prikazane karte koje je korisnik skupio. Karte je zatim moguće poredati prema kategorijama ili prema jačini.}\\
			\item {Globalnu rang ljestvicu - otvara novi prozor koji prikazuje redne brojeve igrača, njihova korisnička imena i brojeve bodova poredane silazno.}\\
			\item {Pomoć korisniku - otvara prozor putem kojega se korisnici prijavljuju za ulogu kartografa. Osim toga, ovaj prozor nudi mogućnosti prijave da određena lokacija postane nova karta te kontaktiranja administratora ukoliko korisnik ima primjedbi ili prijedloga. }
			\end{packed_item}
		
		{Izbornik se zatvara ponovnim klikom na njegov simbol te se vraća početni zaslon. Igrač zatim svoje stvarno kretanje prati na karti te pokušava obići što više stvarnih lokacija. Kada igrač stigne na neku lokaciju koja je označena kao karta, pojavljuje se poruka koja javlja da korisnik ima novu lokaciju u kolekciji karata te prikazuje atribute te lokacije.}\\ \\ \\
		
		\textbf{\textit{\small2.2.2.2 Borba}}\\
		{   Ukoliko korisnik ugleda na karti u krugu od 50 km drugoga igrača s kojim se želi boriti, pritiskom na drugog igrača pojavljuje se prozor u kojemu onda potvrđuje slanje pozivnice za borbu ili odustaje od izazova. S druge strane, drugome se  igraču prikazuje iskočni prozor s ponudom za borbu i podatcima o protivniku kao što su korisničko ime, broj bodova i trenutačni rang. Tada drugi igrač može prihvatiti ili odbiti borbu. \\ Ako je borba prihvaćena, svaki od igrača naizmjence bira po jednu kategoriju karata u kojoj će se boriti po rundama. Borba se odvija u tri runde, stoga igrač koji inicira borbu može izabrati dvije kategorije, a izazvani igrač samo jednu. Borbu prvi započinje igrač s manjim globalnim rangom. On odabire jednu od svojih karata iz prethodno odabrane kategorije i baca ju. Drugi igrač zatim iz svojega špila također odabire kartu iz te kategorije i baca ju. Jača karta pobjeđuje te sljedeću rundu otvara pobjednik prethodne runde. Ako igra bude odlučena nakon prve dvije runde, do treće ni ne dolazi. Tada pobjednik dobiva broj bodova koji odgovara broju bodova najjače karte, dok se isti taj broj bodova oduzima gubitniku.\\ Prozor s borbom se zatvara kod oba igrača te im se prikazuje ažuriranje njihovih osobnih bodova i bodova karata iz kojega se zatim vraća .\\}
		
		
		\textbf{\textit{\small2.2.2.3 Bodovanje karata}}\\
		{Svaka karta ima broj bodova koji mu daje kategorija kojoj pripada. Što je kategorija rjeđa i teža za posjetiti (npr. vrh planine naspram nekoga grada) to donosi više bodova. Svaka karta osim toga ima i početnih 100 bodova koji se onda smanjuju za 0.25 boda za svaku posjetu toj lokaciji. Minimalni broj tih bodova 0. Ukupan broj bodova neke karte je zbroj tih dvaju vrijednosti.\\ }
		
		
		\begin{tabular}{ll}
			\textbf{Naziv kategorije} & \textbf{Broj bodova} \\
			Grad                      & 25                   \\
			Naselje                   & 30                   \\
			Umjetnička instalacija    & 35                   \\
			Vrh planine               & 40                   \\
				&                     
		\end{tabular}
	
		
		\textbf{\textit{\\ \\ \\ \\ 2.2.3 Održavanje sustava}}\\
		\textbf{\textit{\\ \small2.2.3.1 Administratori}}\\
		{Administratori imaju sve ovlasti kao i ostali registrirani korisnici uz još neke dodatne. Oni imaju pristup bazi podataka i mogućnost uređivanja profila korisnika. Administratori također pregledavaju prijave za kartografe i odobravaju ih ili odbijaju. Osim toga, ukoliko korisnici pošalju pitanja unutar prozora za pomoć, oni na njih odgovaraju.\\}
		
		\textbf{\textit{\\ \small2.2.3.2 Kartografi}}\\
		{Ako korisniku administrator odobri status kartografa, prije nego što počne koristiti svoje kartografske ovlasti, on mora u svoje korisničke podatke dodati IBAN računa za uplatu i fotografiju osobne iskaznice. Kartografi su osobe zadužene za nadopunjavanje baze podataka s lokacijama koje su igrači prijavili za moguće proširenje. Njima se u padajućem izborniku pojavljuje posebni gumb pomoću kojega se otvara prozor s prijedlozima za nove lokacije. Osim toga, prijedlozi im se prikazuju na karti, a oni ih mogu odbiti, potvrditi, urediti ili označiti da je potrebna potvrda s terena. Kartograf za one prijave za koje je potrebno potvrđivanje s terena može označiti da će ih osobno doći provjeriti, a sustav mu preko vanjskog servisa OSRM dohvaća najbliži put do lokacije.}
		
		
		\eject
	